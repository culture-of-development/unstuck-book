\documentclass[]{book}
\usepackage{lmodern}
\usepackage{amssymb,amsmath}
\usepackage{ifxetex,ifluatex}
\usepackage{fixltx2e} % provides \textsubscript
\ifnum 0\ifxetex 1\fi\ifluatex 1\fi=0 % if pdftex
  \usepackage[T1]{fontenc}
  \usepackage[utf8]{inputenc}
\else % if luatex or xelatex
  \ifxetex
    \usepackage{mathspec}
  \else
    \usepackage{fontspec}
  \fi
  \defaultfontfeatures{Ligatures=TeX,Scale=MatchLowercase}
\fi
% use upquote if available, for straight quotes in verbatim environments
\IfFileExists{upquote.sty}{\usepackage{upquote}}{}
% use microtype if available
\IfFileExists{microtype.sty}{%
\usepackage[]{microtype}
\UseMicrotypeSet[protrusion]{basicmath} % disable protrusion for tt fonts
}{}
\PassOptionsToPackage{hyphens}{url} % url is loaded by hyperref
\usepackage[unicode=true]{hyperref}
\hypersetup{
            pdftitle={UNSTUCK},
            pdfauthor={Nick Larsen},
            pdfborder={0 0 0},
            breaklinks=true}
\urlstyle{same}  % don't use monospace font for urls
\usepackage{natbib}
\bibliographystyle{apalike}
\usepackage{longtable,booktabs}
% Fix footnotes in tables (requires footnote package)
\IfFileExists{footnote.sty}{\usepackage{footnote}\makesavenoteenv{long table}}{}
\usepackage{graphicx,grffile}
\makeatletter
\def\maxwidth{\ifdim\Gin@nat@width>\linewidth\linewidth\else\Gin@nat@width\fi}
\def\maxheight{\ifdim\Gin@nat@height>\textheight\textheight\else\Gin@nat@height\fi}
\makeatother
% Scale images if necessary, so that they will not overflow the page
% margins by default, and it is still possible to overwrite the defaults
% using explicit options in \includegraphics[width, height, ...]{}
\setkeys{Gin}{width=\maxwidth,height=\maxheight,keepaspectratio}
\IfFileExists{parskip.sty}{%
\usepackage{parskip}
}{% else
\setlength{\parindent}{0pt}
\setlength{\parskip}{6pt plus 2pt minus 1pt}
}
\setlength{\emergencystretch}{3em}  % prevent overfull lines
\providecommand{\tightlist}{%
  \setlength{\itemsep}{0pt}\setlength{\parskip}{0pt}}
\setcounter{secnumdepth}{5}
% Redefines (sub)paragraphs to behave more like sections
\ifx\paragraph\undefined\else
\let\oldparagraph\paragraph
\renewcommand{\paragraph}[1]{\oldparagraph{#1}\mbox{}}
\fi
\ifx\subparagraph\undefined\else
\let\oldsubparagraph\subparagraph
\renewcommand{\subparagraph}[1]{\oldsubparagraph{#1}\mbox{}}
\fi

% set default figure placement to htbp
\makeatletter
\def\fps@figure{htbp}
\makeatother

\usepackage{booktabs}

\title{UNSTUCK}
\author{Nick Larsen}
\date{2020-03-06}

\begin{document}
\maketitle

{
\setcounter{tocdepth}{1}
\tableofcontents
}
\chapter*{Motivation}\label{motivation}
\addcontentsline{toc}{chapter}{Motivation}

Hello and welcome! First off, a story.

I'm 16 years old, sitting in my introduction to physics class and we're
learning about how if you throw a ball it goes up and down in the shape
of a parabola. We look at a fancy slow motion video that shows the
location of the ball every few frames, then traces it out, then it shows
the equation.

\[d = \frac{1}{2}at^2\]

This equation is really simple to use, it's a couple of basic
mathematical operations and as long as you know that the \(a\) in there
is a constant you can look up, you're good to go. And the best part is,
it just works. You can use this simple equation to \emph{very}
accurately predict how long you have remaining without pain when you
drop a hammer from your hand that's destined for your foot.

I always kinda got into stuff like this, so when I got home I decided
I'd do some experiments. I got one of my friends who had a video camera
to bring it over, and we started dropping hammers (but not on each
other's feet). We decided to be somewhat scientific about it, so we
dropped the thing a few times to see how it fell, then we measured out
exactly 1 foot, all the way up to as high as we could hold which was 7
feet and we recorded each height 3 or so times.

Our teacher had told us that \(a = 32\frac{ft}{s}\), so we plugged it in
and we decided to try to make it take exactly one second to hit the
ground. Much to our surprise at the time, that means we only had to lift
it up 16 ft; I remember we were both out of our minds because we swore
it would need to be dropped from 32 ft up to stay in the air one second
but that's what the equation says. He went outside; I went up to my
second floor bedroom and hung the measuring tape out of the window. We
marked off 16 ft exactly with a pen on the siding just below the window
sill and dropped that hammer a half dozen times or so, trying our best
to count out the time in our heads each time we dropped it.

We rushed back inside, loaded the video on my computer and started
counting frames for each drop. The best we could find about the camera
was that it had a framerate of 24 frames per second, which I recall
seemed about right. Much to our surprise, the time it took for the
hammer to drop was not super consistent, but when you average out all
the times, sure enough the times were very close to what the equation
predicted. And most of all, we were both proven wrong in our own
predictions about how far a hammer would drop in the first second!

I could barely sleep that night. I was so excited to actually use
something we had learned that I couldn't wait to go back to school and
tell everyone what we had done. When class finally arrives the next day
we got there early and asked as many people as we could how far a hammer
would fall in the first second, and most of them basically told us to
piss off and stop acting like teacher's pets. BUT! a few played along
\emph{and every single one of them said 32 feet}, just like we had
originally thought. Then our moment came, the teacher arrived.

We told the teacher what we had done, and we brought the paper with our
recorded frames for each drop. He took a look over everything and said
it was good work. All of that was well and good, but after all that, and
seeing our measurements, one of the other kids declared absolutely no
way this was right and then doubled down that if we had actually dropped
a hammer from 32 feet high, it would take exactly one second to hit the
ground.

One of the other kids, who was much more willing to believe our data,
chimed in to our defense, ``and how are you going to get 32 feet high
exactly?'', which was actually a really good point seeing as how I was
hanging out a second floor window just to get to 16 feet. Then the clown
popped in with ``yea, why don't you just drop it out of an airplane?''.
The whole class giggled; the teacher started the lesson and that was
that.

Then I went to sleep again, and when I woke up the next morning I had a
thought\ldots{} \emph{what if we did drop the hammer out of an
airplane?} I'm scared of heights so bad I once climbed back down a
crowded ladder off the high dive at the local pool, kids calling me
names each step of the way, to avoid feeling the sudden smack of the
water against my skin. Getting up in an airplane with a window or a door
open was not going to happen. But I thought, what if someone else did?
What would I expect to happen? An airplane is pretty damn high up, so
pretty much no matter what, you'd expect the hammer to be falling pretty
damn fast when it hits the ground.

But isn't the same true for a human if they were to fall out of an
airplane? Some idiots jump out of airplanes, but they don't seem to be
going too fast when they hit the ground (seems like the theory of
natural selection might be a failure as well). Clearly there is more to
this story. Why does this equation work some of the time but not all the
time? I took this question back to my teacher. His response was that
here on Earth we experience drag, a force that opposes gravity in the
case of jumping out of an airplane and that parachutes increase drag
enough to slow a person down to a safe speed. Then he told me we weren't
going to learn about drag for a couple of months but I was free to look
ahead if I wanted to.

I expressed concern that what we were learning wasn't really useful and
he told me that this equation was indeed correct. He told me about a
video from NASA where the astronaut David Scoot performed a similar
experiment,
\href{https://nssdc.gsfc.nasa.gov/planetary/lunar/apollo_15_feather_drop.html}{dropping
a hammer and a feather at the same time on the moon}, and how they
landed at the same time. I couldn't believe it, nor could I find the
video in the school library (before the internet, y'all), but my dad
told me later that evening he remembered seeing something like that on
TV when he was a kid. He was certain he had seen it, and he was happily
ready to confirm that the equation was correct.

I decided to flip ahead a few chapters and see how drag works, and
whoaaaa\ldots{}.

\[drag = coefficient * \frac{density * V^2}{2} * reference\;area\]

I still don't understand all that today. There's another magic
coefficient, only this time you can't just look it up. Density of what?
What the hell is reference area? But good news, I am certain I can
square a velocity and divide it by 2. And then the hardest part of all,
how do use whatever this is with the other equation?

In simple terms, \textbf{I was stuck}.

\chapter*{Part 1: The Framework}\label{part1}
\addcontentsline{toc}{chapter}{Part 1: The Framework}

\begin{center}\rule{0.5\linewidth}{0.5pt}\end{center}

Wherein we breakdown what it is to build software and why it's helpful
to do so.

\chapter{Processes}\label{processes}

When I blah blah blah

\bibliography{book.bib,packages.bib}

\end{document}
